
\chapter{\label{ch:3-lleeg}The effect of constant light on sleep}
\minitoc
\minilof
\newpage
% Why How What?
% look for why in poster from london
% How EEG defined sleep under 14 day of LL
% What increased sleep in the first 12 hours - masking effect?

\section{Introduction}

% why how what??
% Disrupted lighting environment widespread
% Light has many effects
% Know lots about light -> circ rhythms
% Know less about light -> sleep - masking
% sleep is regulated by Process S and Process C
%
% (want to put something in about how circ and sleep disruption is bad)
% Given that light is a widespread issue, and disruption of sleep and
% circadian rhythms are also widespread and very damaging, want to find out
% how light reuglates

Sleep is a reversible behavioral state of perceptual disengagement from and
unresponsiveness to the environment, 
that appears universally required for complex organisms
\cite{meir_h._kryger_principles_2017, cirelli_is_2008}, and extended 
sleep deprivation can eventually lead to death\cite{everson_functional_1995}.
Sleep is regulated by both a homeostatic and a circadian 
system\cite{borbely_two-process_2016}.

\subsection{Homeostatic regulation}
The homeostatic regulation of sleep represents a general
level of sleep that is required for an organism. 
This expresses itself as a set-point of sleep
that is required over a roughly 24 hour period. 
When deprived of sleep, there is a subsequent increase 
in sleep to compensate\cite{porkka-heiskanen_sleep_2013}.
Importantly, sleep has both a duration and an intensity 
dimension\cite{deboer_sleep_2018}.
The intensity of sleep is conceptualised as 
sleep depth, which is behaviourally expressed 
as an increased arousal threshold, and 
is well correlated with power of 
slow waves during NREM sleep\cite{deboer_behavioral_2015}, 
which are increased after sleep deprivation, such that 
despite changes in the duration and timing of sleep, 
the total amount of slow waves accumulated remains 
the same\cite{deboer_sleep_2018, northeast_sleep_2019}.
The mechanisms behind this are currently unclear. 
A variety of putative sleep factors
build up in the brain during sleep and dissipate during wake, 
such as levels of adenosine
\cite{porkka-heiskanen_sleep_2013, schwartz_neurobiology_2015},
which may modulate the network of sleep/wake centres in the brain.
Another potential mechanism is via changes in synaptic strength
that occur during wakefulness and are re-normalised during sleep, 
termed the synaptic homeostasis 
hypothesis\cite{tononi_sleep_2003, vyazovskiy_sleep_2015}.
Importantly, the accumulation of homeostatic sleep need can be affected
by waking experience\cite{fisher_stereotypic_2016},
and there is evidence that 
the slow wave activity (SWA)
increase in response to sleep deprivation is larger 
in darkness than light in nocturnal rodents
\cite{huber_exploratory_2007, vyazovskiy_sleep_2007}, however
this may be due to differences in waking behaviour rather than a direct
effect of light\cite{huber_exploratory_2007}. 
While light can induce sleep in nocturnal rodents\cite{lupi_acute_2008},
it is unknown whether this alters its homeostatic regulation.


\subsection{Circadian regulation}
The circadian regulation of sleep promotes waking and sleep at 
different parts of the day.
The signal primarily comes from the SCN,
as lesions of the SCN lead to arrhythmic sleep\cite{trachsel_sleep_1992}.
What signal does the SCN provide to segregate sleeping 
across the day this way? 
This is currently unknown. 
The SCN is a group of about 20,000 neurons that each have their
own molecular transcription based clocks, and together produce
rhythms in electrical firing across the day\cite{herzog_regulating_2017,
coomans_detrimental_2013},
though how these molecular rhythms translate to differences in neuronal 
firing is also currently unclear\cite{colwell_linking_2011}.
The SCN can provide a timing signal via both anatomical connections, 
as well as secreted factors, as a transplanted SCN isolated
from anatomical connections can still orchestrate timing in the 
recipient\cite{silver_diffusible_1996}.
Lesions of the SCN can increase the total time asleep in mice, 
squirrel monkeys, and Siberian 
hamsters\cite{easton_suprachiasmatic_2004, edgar_effect_1993, 
larkin_homeostatic_2004}, but not in rats\cite{trachsel_sleep_1992}.
Data in humans shows a drive for wake when sleep pressure is 
highest\cite{dijk_paradoxical_1994}.
Some interpret this as the SCN providing a primarily 
wake-promoting signal, while others suggest the SCN
promotes wake and sleep at different times of day\cite{
mistlberger_circadian_2005, scammell_neural_2017}.
Desynchrony of different parts of the SCN under abnormal T cycles 
show that rhythms of REM sleep are associated with clock
gene expression in the dorsomedial SCN\cite{lee_circadian_2009}.
Whether this is due to changes in firing or other factors is 
unknown. 
The SCN projects to several hypothalamic nuclei, including the sleep
regulating VLPO\cite{watts_efferent_1987, sollars_neurobiology_2015}.
The main hypothalamic projection of the SCN is to the 
subparaventricular zone and dorsomedial nuclei of the hypothalamus.
These are key nodes mediating the circadian control of sleep 
as lesions of the ventral SPZ eliminate circadian rhythms in 
sleep and locomotor activity\cite{lu_contrasting_2001}, 
as do lesions of the DMH\cite{chou_critical_2003}.

\subsection{Direct effects of light}
Light can directly induce sleep in nocturnal rodents\cite{lupi_acute_2008},
though this only occurs at certain circadian phases\cite{muindi_acute_2013}.
How light does this is currently unclear. 
pRGCs in the retina can regulate the direct effects of light on 
sleep separately from their circadian effects\cite{rupp_distinct_2019}.
Although they also appear to regulate baseline levels of 
homeostatic sleep pressure and response to sleep 
deprivation\cite{tsai_melanopsin_2009}.
pRGCs project widely throughout the brain, but importantly
project to both the VLPO and SPZ, areas important 
for sleep regulation\cite{muindi_retino-hypothalamic_2014, 
lu_contrasting_2001}.

\subsection{Constant light and sleep}
Constant light alters sleep timing and distribution, in line 
with the changes in their internal period\cite{eastman_circadian_1983}.
The literature on constant light is currently mixed. 
Some reports show increased total sleep, NREM and REM sleep, 
and decreased wake in rats\cite{borbely_circadian_1978, lobo_hypoprolactinemic_1999},
others show increased NREM but decreased REM sleep 
in rats\cite{stephenson_sleep-wake_2012}, another 
report shows a trend to similar findings in mice but
this is not consistent across strains\cite{mitler_sleep_1977}, 
and others show decreased total sleep, NREM and 
REM in rats\cite{fishman_rem_1972}.
A final study failed to show any changes in duration
of NREM or REM sleep in rats, but used 
very low light levels\cite{ikeda_continuous_2000}.
Differences in the lights used for housing may explain the 
differences reported, with some using incandescent\cite{fishman_rem_1972},
others fluorescent\cite{mitler_sleep_1977}, and others 
neon lights\cite{borbely_circadian_1978}, which all produce different
spectra, and therefore lead to different activation of melanopsin 
even at the same lux levels\cite{peirson_light_2018}.
The most consistent finding is an increase in NREM sleep, whereas
the effect on REM sleep is less clear.
Furthermore, while these reports have all investigated the time in each 
state, less focus has been paid to the spectrum.
Sleep has both a time and an intensity component, therefore it 
is crucial to investigate the effect of constant light on the 
spectrum of sleep. 

\subsection{Aim}
% how does it then cause sleep?
Overall, light regulates sleep through a complex interaction of homeostatic,
circadian, and direct effects of light, which are difficult to disentangle.
Constant light causes disruption of the circadian system, see chapter 2, as
well as providing light at all times of the circadian cycle.
We therefore set out to examine the effects of constant 
light on the regulation of both sleep timing and intensity.


% is it all because we used relatively low levels of light so -> less stress
% arousal activation?

\section{Methods}

\subsection{Animals}
All procedures were performed in accordance with the United Kingdom Animals
(Scientific Procedures) Act 1986, under project licence P828B64BC,
and personal licence I14C9C2DE.
All procedures complied with the University of Oxford Policy on the use of
animals in research.
Young male C57BL/6N mice were obtained from Envigo (Alconbury UK).
All mice
were 8 weeks old at the time of surgery.
A total of 10 animals were surgically implanted with EEG/EMG 
recording equipment, with the aim
of obtaining recordings from 8 mice, which was the limit of what could be
obtained using the equipment available.
Unfortunately due to technical issues, only 7 animals were recorded for the
experiment.
Animals were group housed until they underwent surgery, when they were singly
housed in individual ventilated cages, with \emph{ad libitum} access to food
and water.
2 weeks after surgery animals were transferred to custom made individual
plexiglass experimental cages (20.3 x 32 x 35 cm) where they were 
individually housed
throughout the experiment with \emph{ad libitum} access to food and water.
Experimental cages were kept in a ventilated sound-attenuated Faraday chamber
(campden instruments, 2 cages per chamber), under white LED lighting 
at 100--120 lux at cage bottom height while the lights were switched on.
Room temperature and relative humidity were maintained at 22 +/-1C and
50+/-20\% respectively.

\subsection{Experimental design}
\begin{landscape}
\begin{figure}[p]
    \centering
    \makebox[\textwidth]{
	    \includegraphics[width=0.6\paperwidth]{03_lleeg/02_figures/02_sepfigs/01_fig1}
        }
\caption[Experimental Design]{
Experimental Design of the study.
A. A timeline of the experiment.
We surgically implanted the EEG electrodes, before allowing the animals to
recover at least 14 days in standard 12:12 conditions.
We then moved the animals to the experimental chambers, and after 2 days of
habituation, started baseline recordings.
We then turned on the lights constantly for 14 days.
We then returned the animals to 12:12 LD afterwards, by turning the lights
off halfway through day 15.
We recorded two days of 12:12 LD, before sacrificing the animals.
B. Shows the PIR based activity for the time EEG was recorded for a single
animal.
Shown as a double plotted actogram where blue indicates activity, grey
indicates lights off.
We see a prolonged period free running in constant light as we expected,
indicating that the LL worked and the animals did not entrain to any
alternative zeitgebers.
C. EEG/EMG variance double plotted similarly to B for a single animal.
This shows a crude overview of sleep/wake behaviour, when EEG variance is
high and EMG variance is low, this indicates probable sleep states, when EEG
variance is low and EMG variance is high, this indicates probable wake states.
Here we see that the animals also free run in their patterns on the EEG/EMG.
Furthermore, the EEG variance is high during times when the PIR activity is
low, indicating that the animals are likely asleep not just immobile.
}
    \label{fig:design}
\end{figure}
\end{landscape}

The experimental design is outlined in figure \ref{fig:design}.
Animals underwent surgical implantation of recording devices, then were
allowed to recover for 2 weeks before being transferred to recording chambers.
Initially, animals were housed under a standard 12:12h LD cycle for 
habituation and baseline recordings.
Animals were housed in the recording chambers for 2 days to habituate to the
recording cables before 2 days of baseline recordings were obtained.
After the baseline recordings, animals underwent 14 days of constant light
(LL), before being transferred back to 2 days of 12:12 LD.
All animals were culled at the end of the experiment.
EEG and EMG were recorded during baseline, LL, and return to LD.
The locomotor activity of the animals was also recorded using the COMPASS
system\cite{brown_compass_2017}.
Mice were checked at randomly selected times during the working day (0700h to
1600) to ensure there was no signal they could entrain to.

\subsection{Electroencephalogram recordings}

\begin{figure}[h]
    \centering
    \makebox{
        \includegraphics[width=\textwidth]{03_lleeg/02_figures/02_sepfigs/06_methodsfig}}
\caption[Methods]{
Methods used for recording EEG.
A. Schematic representation of a mouse skull and electrode placement.
B. Representative traces of NREM sleep, REM sleep and Wake, which were used
for manual scoring of vigilance states.
}
\label{fig:methods}
\end{figure}

Surgical recordings were performed as previously described\cite{fisher_stereotypic_2016}.
Surgical procedures were performed under aseptic conditions under isofluorane
anaesthesia (3-5\% induction, 1-2\% maintenance).
Before surgery, animals were provided with analgesia (meloxicam 1--2mg/kg SC,
buprenorphine 0.08mg/kg SC).
EEG screw electrodes were implanted unilaterally in the right frontal
(anterior-posterior 2mm, mediolateral 2mm), and occipital (anteroposterior 
	-3.5mm, mediolateral 2.5mm) cortices, coordinates given relative to bregma.
A reference electrode was implanted above the cerebellum in the midline, and
a ground screw was placed contralaterally to the occipital electrode.
Single stranded stainless steel wires were implanted on either side of the
nuchal muscle to record electromyogram.
All electrodes were attached to a custom-made head-mount (Pinnacle
Technology), and attached to the skull using dental acrylic.
Post-surgery mice were monitored closely and provided with analgesia as
necessary (meloxicam 1--2mg/kg sid for 3 days, buprenorphine 0.08mg/kg PRN),
until they returned to normal.

\subsection{Signal processing}
A Tucker-Davis technologies Multichannel Neurophysiology Recording System was
used for data acquisition.
Cortical EEG was recorded from both the frontal and occipital derivations.
EEG and EMG were filtered between 0.1Hz and 100Hz, amplified (PZ5
NeuroDigitizer preamplifier, Tucker-Davis Technologies), and stored on a
local computer at a sampling rate of 256.9 Hz.
EEG and EMG were resampled offline at 256Hz, and converted into European
Data Fomat using custom MATLAB scripts available at \url{https://github
.com/A-Fisk/VVlab_Scripts}, and the open source Neurotraces software.
Recordings were split into 4 second epochs, and vigilance states scored
offline by manual inspection using SleepSign for animals
\url{http://www.sleepsign.com/}.
Vigilance states were classified as wake (low-voltage, high-frequency EEG,
with high or phasic EMG activity), NREM (EEG slow waves with high amplitude
and low frequency), or REM sleep (low-voltage, high-frequency EEG with low
EMG activity), based upon visual inspection of the two EEG channels and the
EMG channel.
Artifacts, caused by movement in any channel, were also scored to facilitate
analysis.
The data were then exported as power spectra per epoch, computed by FFT using a
Hanning window.


\subsection{Data analysis}
All analysis was written using custom scripts available online at
\url{https://github.com/A-Fisk/sleepPy}, \url{https://github
.com/A-Fisk/03_lleeg}, using a combination of MATLAB and Python.
Baseline days were the last day of normal 12:12 LD, defined from the start of
lights on (ZT0), then at least the first two days of LL were scored for all
animals.
Statistics were performed using the pingouin package
\url{https://pingouin-stats.org/api.html}.
Mixed ANOVAs were performed, followed by post-hoc tukey testing as stated in
results.


\section{Results}


% See prolonged period, reduced activity, looks like induced sleep

% See more sleep overall, delayed activity episode.

% Regulation is intact, duration increased but intensity compensated.

% Cannot find any differences in that sleep

\subsection{Overview of constant light effects on activity and sleep}
% See prolonged period, reduced activity, looks like induced sleep
We recorded EEG from 7 animals under a baseline day of 12:12 LD, followed by
14 days of continous light exposure (LL), and then a further 48 hours of LD.
We also recorded PIR based activity from animals at the same time.
A representative example of both is shown in figure \ref{fig:design}.
In the PIR activity, we see a prolonged period as the animals free run under
constant light, and a reduced level of activity, which is not rescued by the
return to LD.
In figure \ref{fig:design} C, we see EEG and EMG variance as a proxy marker
of sleep and wake.
When EEG variance is high and EMG variance is low, the animal is likely to be
in NREM sleep due to the presence of slow waves and behavioural immobility;
when EEG variance is low and EMG variance is high, the animal
is likely to be in active wake.
We see a similar pattern in the variance profile, with a shorter activity
period that free runs throughout the LL period.
In this example at least, when the PIR activity is reduced, the EEG variance
stays high, indicating that this animal is likely asleep instead of
awake but inactive.

\subsection{Constant light increases sleep}
\begin{landscape}
\begin{figure}[p]
    \centering
    \makebox[\textwidth]{
	    \includegraphics[width=0.8\paperwidth]{03_lleeg/02_figures/02_sepfigs/02_fig2}
        }
\caption[Hypnograms]{
Hypnograms and time in different stages of sleep in the first two days in LL.
A, B, C.
Hypnograms showing slow wave activity for the Baseline, Day 1, and day 2 in LL
respectively,
colour coded for vigilance states NREM (blue), REM (orange), wake (green).
Grey indicates the dark phase, and the dashed line indicates the start of the
subjective dark period.
D, E, F, G.
Total amount of Sleep, NREM, REM, and wake respectively, split into total for
24 hours, and the subjective light and subjective dark phases.
Differences of p$<$0.05 from the baseline day are indicated with a 
line above the plot, orange for day 1, green for day 2.
}
    \label{fig:hypnogram}
\end{figure}
\end{landscape}

% See more sleep overall, delayed activity episode.
We next looked in more detail at the sleep by manually scoring the 
baseline day and the first two days of LL for vigilance states. 
Time constraints prevented us from scoring further days under LL.
To see an overview of the effect on sleep distribution and 
intensity, we plooted a hypnogram of SWA over the first two
days of LL, coloured by vigilance state, an example of which is shown 
in figure \ref{fig:hypnogram} A, B, and C.
In the baseline day we see the expected reduction in SWA during the light
phase as the animal is asleep, followed by long activity episodes once the
lights turn off, and increased SWA in the subsequent sleep episodes.
In both B and C, we see a similar pattern in the subjective light phase, 
with reducing SWA during NREM sleep in the subjective light phase, 
however when the lights fail to turn off 
the animals then remain intermittently asleep for the next 6 hours, 
before having a major activity episode followed by increased SWA.

To quantify this increase in sleep, we calculated the time spent in each
state, sleep, NREM sleep, REM sleep, and Wake, across each 24 hours, as well
as during the subjective light and dark phases, shown in figure 
\ref{fig:hypnogram} D, E, F, G.
We asked whether the amount of time in each stage was affected by the day
using a repeated measure ANOVA, followed by pairwise post-hoc tukey testing,
for the total 24 hours, subjective light and subjective dark phase.

For total sleep, we find an increase on day 1 and day 2 of 1.57 and 1.19
hours respectively (ANOVA, F(2, 12)=16.418, p=0.00037, post-hoc tukey
difference from baseline p$<$0.05 both).
In the subjective light phase, we do not see a difference between the days
(ANOVA, F(2, 12)=3.55, p=0.06), however in the subjective dark phase, there
is a difference, with day 1 and day 2 increased by 1.57 and 1.48 hours
respectivly
(ANOVA, F(2, 12)=26.31, p=4.1e-5, post-hoc tukey, p$<$0.05 both).

For NREM sleep we see a similar pattern, with increased across 24 hours of 1
.39 and 1.09 hours for day 1 and 2 respectively (ANOVA, F(2, 12)=11.02, p=0
.002, post-hoc p$<$0.05 both), no change during the subjective light (ANOVA, F
(2, 12)=1.27, p=0.32), and an increase of 1.39 and 1.29 for day 1 and day 2
in the subjective dark phase (ANOVA, F(2, 12)=26.15, p=4.22e-5, post-hoc p$<$0
.05 both).

When looking just at REM sleep, we do not detect an increase across 24 hours
(ANOVA, F(2, 12)=3.57, p=0.061), which is also true for the subjective light
phase (ANOVA, F(2, 12)=1.04, p=0.38), however we do see an increase of 0.18
and 0.19 hours on day 1 and day 2 when
looking just at the subjective dark phase (ANOVA, F(2, 12)=8.74, p=0.0046,
post-hoc p$<$0.05 both).

Finally we looked at total time spent awake, which as expected, shows the
inverse pattern to total sleep time, with decreases of 1.57 and 1.05 hours
across the whole 24 hours (ANOVA, F(2, 12)=14.84, p=0.00057, post-hoc p$<$0.05
both).
We do detect a difference in the subjective light period (ANOVA, F(2, 12)=6
.99, p=0.0097), but post-hoc testing failed to detect differences between the
days (p$>$0.05 both days).
In the subjective dark period, we see decreases of 1.58 and 1.47 hours on day
1 and 2 respectively (ANOVA, F(2, 12)=26.26, p=4.14e-5, post-hoc p$<$0.05 both).

We therefore find increases in total sleep, NREM sleep, and REM sleep, and a
decrease in time spent awake in response short term exposure to constant
light, mainly localised to the subjective dark period when the light
exposure is abnormal.

\subsection{Increased sleep is regulated homeostatically}
We have so far shown that constant light induces sleep in the first two days
of exposure.
Next we asked how this sleep is regulated.
Sleep is regulated in a circadian and a homeostatic manner, the circadian
system leads to enhanced tendency to sleep at certain times of day, and the
homeostatic regulation leads to a build up of sleep pressure which is
dissipated by sleep.
Homeostatic regulation incorporates two aspects of sleep, sleep time and
sleep intensity.
Sleep time is easily measured by scoring of the EEG, sleep intensity is
measured by the power in the delta frequency band (0.5-4Hz), termed slow wave
activity as this is mainly compromised of the characteristic slow waves seen in
NREM sleep.

\begin{landscape}
\begin{figure}[p]
	\centering
    \makebox[\textwidth]{
	    \includegraphics[width=0.8\paperwidth]{03_lleeg/02_figures/02_sepfigs/03_fig3}
       }
\caption[Sleep Distribution]{
Constant light increases sleep and slow wave energy in the first 5 hours of
the subjective dark phase.
Baseline day in blue, day one in orange, day 2 in green.
Lines above indicate post-hoc testing differences where
p$<$0.05 from baseline for day 1, orange, and day 2, green, unless
otherwise stated.

A. Total sleep per hour. 
B. Cumulative NREM sleep, black line above indicates where day 1 
and day 2 are different from baseline p$<$0.05.
C. SWA per hour.
D. Cumulative SWE.
}
    \label{fig:cumulative}
\end{figure}
\end{landscape}


We therefore asked about the sleep timing and sleep intensity. 
Figure \ref{fig:cumulative} A looks at total sleep duration.
We asked whether the percentage of each hour spent asleep across each hour,
was different for each day, using a two-way repeated measures ANOVA of sleep
\% \~hour*day, taking into account each animal, with pairwise post-hoc tukey
testing for each hour.
We find that there is a difference between the days and hours (ANOVA,
interaction F(48, 288)=5.51, p=9.71e-21).
Post-hoc testing shows that this is due to day 1 differing from baseline at
hours 12-17, and day 2 at hours 11-17.
Day 1 had isolated differences at hours 2 and 22, while day 2 was
different at hour 7 (post-hoc tukey p$<$0.05 all).
The majority of the difference therefore occurs at the transition to the
subjective dark phase.

Given that sleep pressure is dissipated during NREM sleep, we therefore
plotted the cumulative NREM sleep time over each day, as shown in figure
\ref{fig:cumulative} B.
We have already shown in figure ~\ref{fig:hypnogram} an increase in NREM sleep
over 24 hours, and we see that here too as the total NREM sleep is
increased.
However, when we look at the timecourse, we see that the animals continue to
gain NREM sleep at the start of the subjective dark period, and continue to
do so throughout the subjective dark to lead to the higher total amount of
sleep.
This leads to differences from baseline for both day 1 and day 2 at hours
14-23 (pairwise post-hoc tukey p$<$0.05 all).

We next looked at the intensity of sleep, by looking at relative SWA.
The timecourse of SWA is shown in figure \ref{fig:cumulative} C.
We calculated the mean SWA during NREM as a \% of the mean SWA for the baseline
day for each animal, excluding any hours which had $<$5 minutes of NREM sleep.
On the baseline day, we see the expected pattern of a decrease during the
light phase as animals sleep and dissipate sleep pressure, with a rise at
the start of the dark phase, which is maintained during the dark phase as
animals are awake and accumulating sleep pressure.
On day 1 and day 2, we also see a normal decrease in SWA during the
subjective light phase, however at the start of the subjective dark phase the
SWA remains low, increasing on day 1 faster than on day 2, then increasing once
animals wake up and start accumulating sleep pressure again.
We then asked whether the SWA was different for each hour and day, using a
two-way ANOVA followed by pairwise tukey post-hoc testing.
We were unable to use a repeated measures ANOVA, as removing hours with $<$5
minutes of NREM sleep led to missing values.
We find a difference between the days and hours (ANOVA, interaction F(2, 48)
=1.82, p=0.0013), indicating the time-course of SWA is not the same for each
day.
Post-hoc testing showed that on day 1 there are differences from baseline at
hour 8, and 14, and on day 2 there are differences at hour 14, 15, and 22
(post-hoc tukey p$<$0.05 all).
The large error bars indicate high variance at each time point, but the 
consistent finding is that at the start of the subjective dark phase,
while animals are still asleep under LL, their SWA is lower than during
the baseline day. 

Finally we asked whether the SWA was regulated homeostatically.
If it is homeostatically regulated then there is a set amount of
slow waves it should get across the 24 hours, and despite increases in the
duration of NREM sleep the decreased intensity of that sleep means they
return to the same set point.
We therefore plotted the total SWA during NREM sleep
accumulated over the day, termed
slow wave energy (SWE), as shown in
figure \ref{fig:cumulative}D, as a percentage of the maximum value each animal
reached at the end of the baseline day.
We asked whether the total amount of SWE the animals experienced across 24
hours, was altered by the days, using a repeated measures one-way ANOVA on the
maximum values for each day.
We find that there is no difference between the maximum SWE across the days
(ANOVA, F(2, 10)=1.77, p=0.22).
When looking at the cumulative graph of SWE in figure \ref{fig:cumulative}
D, we see that the intensity changes compensated for the increased
duration of NREM sleep during the subjective dark phase.

We therefore find that constant light induces sleep in the first 5 hours of
the subjective dark phase, but there is no alteration to the homeostatic
regulation of sleep across 24 hours.


\subsection{Architecture of light induced sleep}

\begin{landscape}
\begin{figure}[p]
    \centering
    \makebox[\textwidth]{
	    \includegraphics[width=0.8\paperwidth]{03_lleeg/02_figures/02_sepfigs/04_fig4}
       }
       \caption[Effect of LL on NREM power spectra and architectrue of sleep]{
	       Effect of LL on NREM power spectra and architectrue of sleep.
Baseline day in blue, day one in orange, day 2 in green.
Lines above indicate post-hoc testing differences where
p$<$0.05 from baseline for day 1, orange, and day 2, green, unless
otherwise stated.
A. Spectrum of NREM sleep during CT 12-17.
B. Number of NREM episodes per hour. 
C. Mean duration of NREM episodes per hour. 
}
    \label{fig:spectra}
\end{figure}
\end{landscape}

Finally, we asked whether we could detect any changes in the sleep that was
induced between hours 12-17.

First, we looked at the spectrum of NREM sleep during that time period, as
shown in figure \ref{fig:spectra}A.
We used a two-way repeated measures ANOVA to ask if the power was affected by
the day at each frequency bin.
We found no difference between the different days across frequencies (ANOVA,
interaction F(160, 810)=-5.07, p=1).
While in figure\ref{fig:cumulative}C, we found differences in relative
SWA at this time,
here across the whole absolute spectrum of NREM sleep during 
the 6 hours where sleep was induced, we do not see a difference.
Referring to figure\ref{fig:spectra}A, there is a trend to decreased
power in the SWA frequencies (0.5-4Hz), 
but we failed to see an effect here, likely due to no change in 
the remaining frequencies.

Next we looked to see if the sleep was fragmented, by looking at the number
and duration of NREM episodes, as shown in figure
\ref{fig:spectra} B and C respectively.
We asked whether the number and duration was affected by day at 
different times
using a two-way repeated measures ANOVA, with pairwise post-hoc tukey testing
at each time.
We find that the number of NREM episodes is affected (ANOVA, interaction F(46,
276)=2.84, p=7.78e-8),
on day 1 and 2, and the hours 12-16
are increased compared to baseline (post-hoc tukey p$<$0.05 all).
We find that the mean duration of NREM episodes is not affected 
differently by day, or by hour (ANOVA,
interaction F(46, 230)=0.90, p=0.65, time F(23, 115)=0.43, p=0.99).

We therefore find that light induces sleep by increasing the number 
of NREM episodes during the first 5 hours of the subjective dark period,
but they are of similar duration to during the subjective
light period, and the overall spectrum is also
comparable.



\section{Discussion}
% Key findings
% potential mechanisms
% implications
% strengths/limitations
% Future studies
% conclusions
\subsection{Key findings}
Here we show that housing mice under constant light induces 
both NREM and REM sleep over the first two days, but that sleep 
is regulated homeostatically and appears similar to sleep during the 
subjetive light phase.
% Increased NREM, REM sleep, decreased wake.
\subsubsection{Sleep induction}
We show in figure \ref{fig:hypnogram}, that there is an increase of 
1.5 hours of total sleep in the first two days of LL, composed of 
increases in both NREM and REM sleep during the subjective dark phase. 
This agrees with previous findings reporting increases in both 
NREM and REM sleep in rats\cite{borbely_circadian_1978, lobo_hypoprolactinemic_1999},
however it contradicts findings of decreased REM sleep
in rats and mice\cite{mitler_sleep_1977, stephenson_sleep-wake_2012},
and decreased NREM and REM sleep in rats\cite{fishman_rem_1972}.
We did not find an effect on REM sleep when looking across the 
entire 24 hours, but only found an increase in REM sleep during the 
subjective dark phase.
We used LED lights at 100-120 lux, while previous reports have used a 
variety of different light sources and not reported their light levels.
Furthermore, previous reports have used different rat strains,
and this is the first report of the effect of constant light on 
sleep in mice to show increased sleep.
% Regulated homeostatically.
\subsubsection{Regulation of induced sleep}
In figure \ref{fig:cumulative},
we show that although the duration of NREM sleep is increased 
during the first two days under LL, 
the intensity of sleep compensates for the increased 
duration.
This leads to the total amount of SWE accumulated over 24 hours staying 
constant.
This is similar to other reports where altering the daily 
organisation of sleep-wake behaviours did not alter the total
amount of SWE accumulated over the day in mice\cite{northeast_sleep_2019},
although interestingly that report showed that total SWE was 
altered for the first few days under restricted feeding, and only
demonstrated equivalence to baseline after 7 days of habituation.
This is in contrast to our findings where we see immediate changes 
in the first two days of exposure to the light cycle.

% No differences seen in that sleep
\subsubsection{Architecture of induced sleep}
Since we found LL induced sleep between CT 12 and CT 17, but this was 
regulated homeostatically to not alter the total amount of
SWE accumulated over the day, we then asked if there was any 
difference in the sleep induced during this time period. 

We compared the spectra of NREM sleep between the baseline day, and 
the subsequent days during CT 12-17. 
We found no differences in the absolute spectrum across days.
This is different to the findings of figure \ref{fig:cumulative}, which 
found changes in relative SWA at this time. 
We are comparing the entire absolute spectrum in figure \ref{fig:spectra} A, 
instead of just the subset which makes up SWA. 
Indeed when looking figure \ref{fig:spectra}A, there appears to lower levels 
around the SWA frequency band under LL, but when combined with the rest of the 
frequencies in the spectrum we did find evidence of an effect.
We took CT 12-17 for the baseline day as the time to compare against,
however this is a time when there is little sleep during the baseline 
day, while the first two days in LL are predominantly sleep. 
This answers the question of what the difference is in sleep between
the three days at this time.
A future comparison that may be useful would be the last hour of the 
light period on the baseline day, as this is the equivalent of a 
sleep-dominated, but low sleep-pressure period of sleep. 

Finally, we looked at the number and duration of NREM sleep episodes.
We see increased number of NREM episodes compared to baseline at CT 12-16,
but this is roughly the same number of episodes as during the subjective
light phase. 
We see no change in the mean duration NREM episodes under LL.
There is no change in the mean duration of NREM episodes across the 
day either, indicating the change in time in NREM sleep is due to 
changes in the number of episodes, not in their duration.
\newline

Overall, we show that constant light induces ~1.5 hours extra sleep
across the first 5 hours of the subjective dark period.
This extra sleep is homeostatically compensated to allow a normal
amount of SWE accumulation throughout the day, and has similar
architecture to sleep during the subjective light phase.

\subsection{Potential mechanisms}
% reduction of circadian signal
% Direct changes? 
Sleep is affected by circadian regulation, homeostatic regulation,
and direct effects of light. 
Here we outline the evidence of the involvement of each of these 
on the effects we see under constant light.
\subsubsection{Circadian mechanisms}
The locus of the circadian control of sleep and wake,
similar to most other functions, is the SCN\cite{ralph_transplanted_1990, 
scammell_neural_2017, mistlberger_circadian_2005}.
An important output signal is the electrophysiological firing rate, as
optogenetically altering SCN firing rate using 
dopamine receptor\cite{jones_manipulating_2015}, or 
vasoactive intestinal peptide\cite{mazuski_entrainment_2018} 
driven ion channels is enough to entrain locomotor activity
rhythms. 
The SCN projects to the VLPO, as well as SPZ and DMH which
are also crucial for the circadian rhythms of sleep and 
wake\cite{watts_efferent_1987, sollars_neurobiology_2015, 
lu_contrasting_2001, chou_critical_2003}. 
Under constant light, the rhythms of SCN firing become 
flattened in mice\cite{coomans_detrimental_2013}, staying 
above baseline dark-period levels, 
due to reducd synchrony among SCN neurons\cite{hughes_constant_2015,
ohta_constant_2005}.
However, it is known that behavioural state can alter 
the firing rate of the SCN\cite{deboer_sleep_2003}, 
therefore we do not know if the alterations in firing rate seen 
in\cite{coomans_detrimental_2013} are cause or 
consequence of changes in sleep.
Two optogenetic studies caused behavioural delays by increasing
SCN neuronal firing during the subjective 
night, and one caused activity masking similar to light
\cite{jones_manipulating_2015, mazuski_entrainment_2018}.
Therefore the presence of extra light during the start 
of the subjective dark phase may be increasing the SCN firing 
rate which therefore increases sleep, possibly through 
increased stimulation of the VLPO, or reduced wake-promotion
through other projections. 
This model is simplified as there is heterogeneity in the 
firing patterns of neurons within the SCN\cite{mazuski_entrainment_2018},
and there are also other non-neuronal signals which can
entrain rhythms of sleep and wake\cite{silver_diffusible_1996}, 
so it may be more appropriate to say that constant light 
pushes the SCN to a more 'day-like' state, characterised by 
increased firing rates, which promotes sleep. 

\subsubsection{Direct effects of light}
Light directly induces sleep in nocturnal animals 
via melanopsin expressing pRGCs
\cite{lupi_acute_2008, altimus_rods-cones_2008, tsai_melanopsin_2009,
muindi_acute_2013}.
This effect is present during the subjective light and the 
first half of the subjective dark phase\cite{muindi_acute_2013, 
lupi_acute_2008, tsai_melanopsin_2009}, although 
Altimus et al\cite{altimus_rods-cones_2008} failed to show 
any sleep induction effect in mice, but they used a T7 cycle, 
as opposed to T2 or light pulses.
The effects of prior sleep-wake history between
these different light cycles may confound these 
results\cite{muindi_retino-hypothalamic_2014}.
This sleep induction also appears to be capable of 
lasting up to 3 hours\cite{altimus_rods-cones_2008},
and 6 hours of light at ZT 13-19 induces 1 hour of extra
NREM sleep\cite{muindi_acute_2013}.
The mechanisms by which light induces sleep are currently 
unclear. 
pRGCs project both to the SCN and to extra-SCN areas important 
for sleep and wake, including the VLPO and 
lateral hypothalamus 
(LH)\cite{muindi_retino-hypothalamic_2014, mistlberger_circadian_2005}.
These extra-SCN projections appear to be neccesary for the 
acute light-induced sleep\cite{rupp_distinct_2019},
and light pulses induce FOS in the VLPO\cite{lupi_acute_2008, 
tsai_melanopsin_2009}.
This suggests that the increase in sleep we see in the 
first 6 hours of the subjective dark period could be 
due to light activating melanopsin 
expressing pRGCs which stimulate the VLPO and 
induce sleep. 

\subsubsection{Homeostatic regulation of increased sleep}
Sleep is regulated homeostatically, by both time and intensity.
This is represented by SWE, which maintains the same levels 
under LL over 
24 hours despite changes in the time asleep\cite{northeast_sleep_2019}.
It is unlikely that changes in homeostatic sleep need are causing
the increased sleep time, as that sleep 
still has low SWA, as seen in figure\ref{fig:cumulative}.
Therefore we do not find increased sleep need being the cause 
of increased sleep time. 
\newline

Together, the changes we see are most likely due to a combination 
of direct induction of sleep, with a small contribution of an 
altered circadian sleep/wake signal, as we see an immediate
effect.
This is primarily because the increase in NREM sleep that we see 
during the subjective night in the first two days (1.39 and 1.29 hours 
respectively), is slightly larger than the ~1 hour induced by a 6 hour 
light pulse during the subjective dark\cite{muindi_acute_2013}.
We are assuming the difference between these two values is due to 
both the increase in light duration, and changes in the circadian
sleep/wake signal.



\subsection{Implications}
Here we show that constant light induces sleep at the 
start of the subjective dark phase.
We assume this is due to a direct light induced effect, with
a small contribution of a change in the circadian slee/wake signal.
We see no changes in homeostatic sleep pressure and conclude 
that the extra sleep we see is not due to any change in sleep
homeostasis.

Sleep regulation is modelled by by a two-process model\cite{borbely_two-process_2016},
where the homeostatic process S builds up as a function of prior wake 
and dissipates during sleep, while the circadian process C varies
to favour wake and sleep throughout the day. 
While originally formulated for humans, lowering the upper threshold 
for process S accumulation models the polyphasic sleep seen in
rodents\cite{daan_timing_1984}.
Application of this model to SWA in rats suggested that there 
were different dynamics to SWA accumulation during 
light and dark periods\cite{vyazovskiy_sleep_2007}.
Furthermore there appears to be a circadian modulation
of the SWA dynamics in constant darkness\cite{deboer_sleep_2009}. 
Two elaborations of the two process model have been 
published recently.
The first elaborates the parameters predicting SWA to incorporate
recent history, and provides quantitative predictions of 
SWA across light, dark, and different waking states
\cite{guillaumin_cortical_2018}.
This model provides quantitative predictions of SWA, which are highly 
accurate to empirical observations, however it does not directly 
predict state or transitions.
Another recent model attempted to address this limitation by 
implementing a stochastic markov-chain-model to predict 
duration and transition between
episodes of different states\cite{rempe_mathematical_2018}.
This also provided accurate predictions, however required parameter 
tuning based on the waking state.
This latter model had varied state transition and durations
based upon Process S and Process C. 
Our interpretation of our findings, is that process S is not affected
and alterations in process C provide a minimal input to the changes
we see.
This second point is a major assumption, but fits with the previously
published literature\cite{coomans_detrimental_2013}.
Furthermore how light induced sleep fits with the two-process model
is still an important outstanding question, given recent work
showing that this requires extra-SCN projections of pRGCs, 
i.e. the SCN does not directly mediate the direct sleep induction by 
light\cite{rupp_distinct_2019}.
Assuming that our results indicate differences beyond changes that 
can be modelled by process S or process C, where does this leave our 
mathematical models?
While the sleep induction effect of light has been studied well,
how it fits into a wider sleep regulation framework is less 
clear\cite{muindi_retino-hypothalamic_2014}.
The Rempe et al model\cite{rempe_mathematical_2018} provides a possible 
formalisation of the changes we see.
The state transitions and 
episode durations are already affected by Process C and Process S, 
and therefore alterations of this stochastic process may be able 
to model the prolongation of sleep we see under LL.
Indeed, in agreement with this,
we see that episode duration of NREM sleep stays similar to 
during the subjective light phase during the first few hours of constant 
light, as seen in figure \ref{fig:spectra}.
In the future, we hope to see whether incorporating light 
environment in some way into these models can capture the changes we see here. 

\subsection{Limitations}

\subsubsection{Circadian vs Direct effects}
% Circadian vs direct effects unknown 
The major limitation of the data presented here is we are unable to 
determine whether the effects are due to changes in the circadian
system or due to the direct effects of light.
We are assuming that they are mainly due to the direct effects 
of light as we see similar sized changes as previously 
reported\cite{muindi_acute_2013}.
In order to address this, we would want to alter one of these processes
while not affecting the other, which is difficult given their interrelated
nature. 
Since extra-SCN projections of pRGCs were found to be neccessary and 
sufficient for the direct effects of light on sleep\cite{rupp_distinct_2019},
performing our LL protocol in mice with either only SCN or extra-SCN pRGC 
projections would indicate the extent of the direct effect of light 
in our findings.
Though this may be complicated by feedback onto the 
circadian system from extra-SCN areas regulated by light.
On the other hand, our hypothesis that LL converts the SCN to a more
day-like functional state partially 
due to desynchrony between SCN neurons, 
suggests that if we could induce this state separately to 
light input then we would observe the SCN contribution to our 
findings. 
This may be approached by using optogenetic approaches to 
alter the functional state of the SCN in
constant darkness\cite{jones_manipulating_2015, mazuski_entrainment_2018}. 
Alternatively, \emph{Vipr2\textsuperscript{-/-}} mice have arrhythmic SCNs similar
to those seen under constant light\cite{hughes_constant_2015, 
ohta_constant_2005}, so their sleep under DD would provide an
insight into the effects of an arrhythmic SCN on sleep. 
Our hypothesis suggests that either of these conditions would only 
marginally alter the amount of sleep per day.
This assumes that the disruption of the circadian system is 
constant and does not alter over time, which may not hold true
\cite{coomans_detrimental_2013}.
Therefore over time, as the circadian system becomes more impaired
its effect may become more pronounced.
We can address this by looking at the effect of LL on sleep after 
an extended period.
We currently have the data to look at this after 14 days of LL, 
however due to practical limitations of manual sleep scoring, 
we were only able to look at the first 48 hours under LL.
To bolster this, we would need to perform this experiment while measuring
the circadian system, such as the firing rate of the SCN. 

\subsubsection{Sleep homeostasis}
% Homeostasis uninterrogated
Our findings rest upon our interpretation that sleep homeostasis
is unchanged by our constant light protocol, 
as the levels of SWA remain low during the induced sleep at the 
start of the subjective dark phase.
This is observational data, and so does not unequivocally show
that sleep homeostasis is unaffected.
Sleep deprivation is commonly
used to investigate changes in sleep 
homeostasis\cite{vyazovskiy_sleep_2007}.
To investigate 
we would need to ask two separate questions; What is the effect
of constant light on the accumulation of homeostatic sleep 
pressure? and what is the effect on the dissipation of 
homeostatic sleep pressure?
For the first question, we would sleep deprive animals
during ZT 12-18, the start of the subjective dark phase,
and look at the SWA accumulation. 
Light environment is a contributor to homeostatic sleep need
\cite{vyazovskiy_sleep_2007, tsai_melanopsin_2009}, therefore 
we may see changes in the build up of SWA.
For the second question, we would sleep deprive animals during  
ZT 6-12, the last 6 hours of the subjective light phase, as we 
could then examine the dissipation of the artificially high SWA.

\subsubsection{Rhythmicity under LL}
Constant light is well known to cause behavioural arrhythmicity,
in proportion to light intensity
\cite{chen_strong_2008, aschoff_exogenous_1960},
due to desynchronisation of neurons in the 
SCN\cite{ohta_constant_2005}.
We used 100-120 lux of white LED light, which did not induce
behavioural arrhythmicity.
This light level is in line with some previous studies 
looking at jetlag\cite{loh_rapid_2010}, but below previous
studies that induced arrhythmicity in constant 
light\cite{ohta_constant_2005, chen_strong_2008}.
While other studies induced arrhytmicity using even
lower light levels than we used here\cite{munoz_long-term_2005,
sudo_constant_2003}, they used different strains of 
mice to us, which are known to affect
light perception and circadian behaviour\cite{peirson_light_2018}.
Therefore we may be seeing minor circadian effects as 
we have not fully disrupted the SCN using LL.
Arguing against this, are our findings from chapter 2, 
which demonstrate that LL still causes major changes to 
circadian activity.
We would have to perform this experiment again under 
light levels that do induce arrhythmicity to address this. 

\subsubsection{Data processing}
We have measured all our markers against clock time, and have not 
corrected for internal circadian time, despite constant light 
inducing free-running.
We did this due to time constraints of our data analysis. 
Since we only examined the first 48 hours of constant light,
we do not believe this is majorly affecting our findings.
The main evidence is from figure \ref{fig:cumulative} which 
shows that even on the second day in LL, there are only
minor deviations from baseline during the subjective light phase. 
As we examine further into LL, we will have to correct for 
circadian phase.

\subsection{Future studies}
Here we show that constant light induces sleep during the 
subjective dark phase in the first two days, and that sleep
is homeostatically compensated.
We assume that the majority of that sleep is due to 
direct effects of light, with some contribution of 
alterations of the circadian system. 
There are several future studies we suggest. 

First, sleep at the end of an extended period of LL, when we assume
the circadian system is more disrupted, will provide information
about the relative contribution of direct effects of light and 
the circadian system. 
We have this data and are working on analysing it. 
Second, modelling how the changes we see fit in with both 
SWA, and state prediction 
models\cite{guillaumin_cortical_2018, rempe_mathematical_2018},
will provide some quantitative insight into how our findings fit 
with our current understanding of sleep regulation. 
Third, performing sleep deprivation experiments to confirm 
that the build up and dissipation of homeostatic sleep pressure 
is normal under constant light.
Finally, I have outlined a range of suggestions for investigating 
the relative contribution of circadian and direct effects of light 
in our protocol in the above sections.
The most productive is likely to be repeating these experiments using 
animals where the direct effect or the circadian effect of sleep
induction by light is removed\cite{rupp_distinct_2019}.
This should be followed by performing this experiment under a range of 
light levels to induce behavioural arrhythmicity, and therefore
see if there is a gradient in the effect of the circadian 
contribution.

\section{Conclusions}
Here we show that constant light induces sleep during the subjective 
dark phase in the first two days, and that sleep is homeostatically
compensated.
We suggest that this is primarily due to a direct effect of light on 
sleep, with a small contribution of an impaired circadian system. 
This effect is not captured by the quantitative models we currenlty
employ of sleep regulation, and it will be of interest to see 
how they can be adapted to fit our findings.
Our findings rely on a number of assumptions that we hope to be able to address
with future studies.



